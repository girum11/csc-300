% CSC 300: Professional Responsibilities
% Dr. Clark Turner

% Two Column Format
\documentclass[11pt]{article}
%this allows us to specify sections to be single or multi column so that things
% like title page and table of contents are single column
\usepackage{multicol}
\usepackage[normalem]{ulem}

\usepackage{setspace}
\usepackage{url}

%%% PAGE DIMENSIONS
\usepackage{geometry} % to change the page dimensions
\geometry{letterpaper}

\begin{document}

\title{\vfill Firesheep: Disclosing Dangers} %\vfill gives us the black space at the top of the page
\author{
By Girum Ibssa\vspace{10pt} \\
CSC 300: Professional Responsibilities\vspace{10pt} \\
Dr. Clark Turner\vspace{10pt} \\
}
%\date{October 22, 2010} %Or use \today for today's Date
\date{\today}

\maketitle

\vfill  %in combinaion with \newpage this forces the abstract to the bottom of the page
\begin{abstract}
%5\%  One or two sentences of context about your issue. Provide your focus question. Mention one or two relevant arguments.  Briefly explain your conclusion and why it is proper ethically. Should be no more than one paragraph (100 words max) \cite{handout}
Firesheep is an extension for the Firefox web browser that wraps Wireshark (an existing piece of session hijacking and packet sniffing software \cite{wireshark}) in a simple GUI \cite{firesheep-source}. Created by Eric Butler and Ian Gallagher, Firesheep's motivation was described by its creators: ``We're bringing up this tired issue to remind people of the risks they face, especially when on open WiFi networks, and to remind companies that they have a responsibility to protect their users. To drive this point home, we are releasing an open source tool at ToorCon 12 which shows you a `buddy list' of people's online accounts being used around you, and lets you simply double-click to hijack them'' \cite{security-now}.

Was it ethical to release this software? Eric stated in his release of Firesheep that ``It's extremely common for websites to protect your password by encrypting the initial login, but surprisingly uncommon for websites to encrypt everything else...This is a widely known problem that has been talked about to death, yet very popular websites continue to fail at protecting their users'' \cite{eric-butler}. Yet the use of Firesheep may be illegal in the US and beyond \cite{illegal-to-use-firesheep}. I will show that Firesheep was indeed ethical to release primarily due to SE Code 1.04, the requirement for Software Engineers to fully disclose any potential software danger to the public \cite{se-code}.

\end{abstract}

\thispagestyle{empty} %remove page number from title page
\newpage


%Create a table of contents with all headings of level 3 and above.
%http://en.wikibooks.org/wiki/LaTeX/Document_Structure#Table_of_contents has
%info on customizing the table of contents
\thispagestyle{empty}  %Remove page number from TOC
\tableofcontents

\newpage

%end the 1 column format


%start 2 column format
\begin{multicols}{2}
%Start numbering first page of content as page 1
\setcounter{page}{1}
%%%%%%%%%%%%%%%%%%%%
%%% Known Facts  %%%
%%%%%%%%%%%%%%%%%%%%
\section{Facts}
% Known facts that are not disputed that lead to your question. Do not judge these facts or make anything like an argument for an answer in here. Just note the facts that give us the general background and end them with the facts leading to the controversy you are interested in. The reader should naturally be asking the question you'll be asking by that point in your paper. In general, attach your facts to a specific case, the more specific and detailed the facts, the better for your analysis. Cite all facts to their sources. \cite{handout}

%\begin{itemize}

%\item General context of the issue
%\item Only facts that  point to the issue/question at hand (you may add additional facts as needed in your analysis that aren't in your facts section)
%\item The facts section should end up leading the reader to the question you are about to ask them (a controversy or a case ending poorly is a great way to do this).
%\item Should not indicate anything about your answer to the question or hint at any particular conclusion.
%\cite{handout}
%\end{itemize}

\subsection{Description of Firesheep}
Eric Butler describes in his blog: ``When logging into a website you usually start by submitting your username and password. The server then checks to see if an account matching this information exists and if so, replies back to you with a `cookie' which is used by your browser for all subsequent requests \cite{eric-butler}.

``It's extremely common for websites to protect your password by encrypting the initial login, but surprisingly uncommon for websites to encrypt everything else. This leaves the cookie (and the user) vulnerable. HTTP session hijacking (sometimes called ``sidejacking'') is when an attacker gets a hold of a user's cookie, allowing them to do anything the user can do on a particular website. On an open wireless network, cookies are basically shouted through the air, making these attacks extremely easy'' \cite{eric-butler}. Butler continues with his description of the ``problem that has been talked about to death'' and how it remains to be solved \cite{eric-butler}. 

Qualified hackers were already able to perform session hijacking like this well before the release of Firesheep, using the program `Wireshark' (which Firesheep is built on top of). \cite{wireshark}.

Firesheep works by allowing users to write custom "handlers" for it to allow it to work on webpages of their choice. Firesheep's source code currently includes "handlers" for several popular websites including: Amazon, Basecamp, bit.ly, Enom, Facebook, FourSquare, Github, Google, Hacker News, Harvest, The New York Times, Pivotal Tracker, Twitter, ToorCon: San Diego, Evernote, Dropbox, Windows Live, Cisco, Slicehost, Gowalla, Flickr and Yahoo \cite{firesheep-source}. 

Some of those websites have fixed the security hole (by authenticating the entire site with HTTPS) that Firesheep relies on to work \cite{butler-fallout}. However, a strong portion of websites simply didn't fix the security flaw, weeks and months after the release of Firesheep \cite{butler-fallout}.


% (Clark told me to take out this section) \subsection{Definition of the ACM Code of Ethics}
%The Association for Computer Machinery defines a Code of Ethics stating that ``software engineers shall commit themselves to making the analysis, specification, design, development, testing and maintenance of software a beneficial and respected profession'' \cite{se-code}. 

% (Clark told me to take out this section) \subsection{Firesheep's adherence to said Code of Ethics}
%Firesheep's release was intended to allow the general public perform session hijacking on suspecting or non-suspecting victims, with consequences that may not be ``beneficial and respected.'' However, Firesheep's release prompted many large software companies to quickly fix the security holes that Firesheep was designed to expose \cite{disconnect-blog}; the companies were slow to fix it before Firesheep's release \cite{disconnect-blog}. From a strictly objective point of view, Firesheep was detrimental to public welfare in the short term and was beneficial to public welfare in the long term.



%%%%%%%%%%%%%%%%%%%%%%%%%
%%% Research Question %%%
%%%%%%%%%%%%%%%%%%%%%%%%%
%One sentence (one question, not compound). Should be focused on a particular case. Should have a determinable yes or no answer that you will draw based on your research.  Should be followed (separately) by a paragraph explaining the importance of this question and its relevance to software engineers (why should we care?). \cite{handout}

\section{Research Question}
% Your research question -- this is the ethical question you are interested in answering. It should be one simple sentence and lead to a yes/no answer. It needs to be very narrowly focused, specific, and not abstract at all. It's best to question a detailed case in the general area of your interest. Open ended questions are very hard to answer. \cite{handout}

%Was it ethical for Eric Butler to release Firesheep to the public, knowing that it could be used for harm in the hands of the wrong individuals?

Was it ethical for Eric Butler to release Firesheep to the general public as a means of forcing websites to improve their flawed security?

\subsection{Relevance}
Firesheep can be downloaded today ({\today}) from GitHub.com, a website hosting free open source programs \cite{firesheep-source}. At its time of release, several websites simply refused to implement HTTPS site-wide, allowing Firesheep to work on a large number of sites \cite{eric-butler}. Today, Firesheep still works on a few sites that haven't switched to using HTTPS site-wide, including the entire Stack Exchange network of websites \cite{stack-exchange}.

Session hijacking allows users of Firesheep to impersonate "logging in" as their victims on the sites that it works on \cite{eric-butler}. The consequences of this can be anything from posting false Facebook status updates to outright deleting a person's Stack Overflow profile. Eric Butler's free software simplifies this process to the point where normally unqualified people may perform these acts, increasing the vulnerability of typical users worldwide.


%%%%%%%%%%%%%%%%%%%%%%%%%
%%% Extant Arguments from External Sources %%%
%%%%%%%%%%%%%%%%%%%%%%%%%

%\section{Arguments For}
%\subsection{Arg 1}
%The first argument for your topic
%\subsection{Arg 2}
%The second argument for your topic...
%\section{Arguments Against}
%\subsection{Arg 1}
%The first argument against your topic.
%\subsection{Arg 2}
%The second argument against your topic...

%the * causes a section with no numbering also doesn't appear in the table of contents
%\subsubsection{Requirements for the Arguments section(s) (from the handout)}
%Summarize the main arguments others have made about the answers to your focus question. Provide the state of research on your focus question. Must be referenced appropriately.  All statements must be a summary of the source's arguments, devoid of your opinions or biases on the issue. Must (separately) cover arguments on both sides of your issue - that is, those that answer your focus question affirmatively and those that answer negatively. \cite{handout}

\section{Extant arguments}
% Extant arguments -- this is where you gather the arguments made by others interested in the same question. No judgments, just repeat their arguments for the answer in the best possible light from the arguer's perspective. Cover both sides of your question (the ``yes'' side and the ``no'' side) to get a complete picture of how others are thinking about it. Do not include any general ethical principles in here unless they are explicitly written up in the arguments. Cite all arguments to their sources. 

\subsection{In Favor of Firesheep's release}
\subsubsection{Code as "free speech"}
Eric Butler himself argues in favor of the release of Firesheep. In his article ``Firesheep, a week later: Ethics and Legality'', Butler states outright that ``it is nobody's business telling you what software you can or cannot run on your own computer'' \cite{butler-week-later}. He defends by saying that code is a form of free speech, and that we have a Constitutional right to free speech \cite{butler-week-later}.

\subsubsection{Mozilla's support of Firesheep}
Mozilla themselves supported Firesheep at its time of release. Firefox (the browser for which Firesheep is an extension for) features an internal blacklist of extensions that it does not allow to work \cite{mozilla-blocklist}. Mozilla, the creators of Firefox, specifically decided not to blacklist Firesheep \cite{no-kill-switch}. Mike Beltzner, director of Firefox, praised its release: ``[Firesheep] demonstrates a security weakness in a number of popular websites, but does not exploit any vulnerability in Firefox or other Web browsers'' \cite{no-kill-switch}.


\subsection{Against Firesheep's release}
\subsubsection{Real-world use of Firesheep may be illegal}
The actual use of Firesheep, however, may be illegal \cite{illegal-to-use-firesheep}. Federal wiretapping laws state that it's not illegal ``to intercept or access an electronic communication made through an electronic communication system that is configured so that such electronic communication is readily accessible to the general public'' \cite{illegal-to-use-firesheep}. However, Jonathan Gordan, partner at Los Angeles law firm Alson and Bird, states that ``when people are accessing their social network [account], they have an expectation that whatever they're doing is governed by the privacy settings in that network'', and that a open Wi-fi network does not qualify as ``readily accessible to the general public'' \cite{illegal-to-use-firesheep}.

\subsubsection{Firesheep desired to violate privacy}
Firesheep is built specifically to allow untrained users to hijack browser sessions from unsuspecting victims \cite{eric-butler}, which directly contradicts Section 1.03 of the Code of Ethics: ``1.03. Approve software only if they have a well-founded belief that it is safe, meets specifications, passes appropriate tests, and does not diminish quality of life, diminish privacy or harm the environment'' \cite{se-code}. 



%%%%%%%%%%%%%%%%
%%% Analysis %%%
%%%%%%%%%%%%%%%%
\section{Analysis}
%\begin{itemize}
%   \item Should start with a paragraph showing why the SE Code applies to your focus
%question.
%   \item Sub-headings to delineate your lines of reasoning are required.
%   \item All arguments must be thoroughly supported by reason and logic.
%   \item All claims must be supported by reputable primary sources and formal data.
%   \item SE Code must be central to the argumentation
%   \begin{itemize}
%      \item You should have 2-4 distinct sections of the SE code utilized in your analysis
%      \begin{itemize}
%         \item If section 1 is central to your argument, it is only one of the code sections covered. Do not rely solely on section one. Ex: 1.01-1.04 will not suffice for all of your SE Code based arguments and citations.
%         \item If discussion about Òpublic goodÓ is used, there must be data to support it. Simply writing Òit benefits the general public because it would make many people happyÓ is insufficient.
%      \end{itemize}
%   \end{itemize}
%   \item Utilitarian and deontological analysis must be present but not be separate sections
%   \item Class reading must be referenced as appropriate (at least one paper must be used as the basis of one of the arguments).
%   \item There should be a clear cohesiveness to the analysis such that each argument logically flows into the next and gently directs the reader toward your conclusion while implicitly providing answers to any doubts they may have through logic and data.
%   \item Opinions > dev/null. \cite{handout}
%\end{itemize}

%Look at Jason Anderson's how to write a term paper (currently linked as the paper template) for information on how to write this section.  An example of possible sections follows
%\subsection{Why the SE Code Applies}
%\subsection{Argument 1}
%\subsubsection{Code principle 1 that applies}
%\subsubsection{Code principle 2 that applies}
%\subsection{Argument 2}
%\subsubsection{Code principle 1 that applies}
%\subsubsection{Code principle 2 that applies}

%\subsubsection{}
%Remember to weave the class papers and other ethical systems arguments in with the se code arguments they shouldn't be separate sections.

\subsection{Why is the SE Code of Ethics applicable to this problem?}
The Code defines software engineers as ``those who contribute by direct participation to the...design, development...of software systems'' \cite{se-code}. Did Eric Butler contribute by ``direct participation'' to the design of some ``software system''?

Firesheep is the name of software written in the C++ programming language that wraps existing packet sniffing software (Wireshark) in an easy to use, one-click GUI extension for the Firefox browser \cite{firesheep-source}. Firesheep is a \uline{software system directly written by Eric Butler}, maintained on his open source GitHub account \cite{firesheep-source}. Eric Butler (the software engineer) has therefore \uline{contributed by direct participation} (programming himself) in the design of a ``software system'' (Firesheep) \cite{firesheep-source}.

Since he contributed by ``direct participation'' to ``the design'' of a ``software system,'' Eric Butler is defined by the ACM Code of Ethics to be a ``software engineer,'' and is therefore bound to adhere to the rules of the Code\cite{se-code}.

% Arugment 1: Disclosure

% TODO: The sub subsections outside of the Analysis should be a lot shorter - they should be strictly definitions, not analysis. Move the existing analysis that I have in those sections to the Analysis section. The Analysis section on this first Argument should be about 1.5 pages at least.

% TODO: Section 4.2.2: "Forcing websites" might not fit in Code 1.04 for this section. Move this to the Analysis section under a subsection "Danger."

\subsection{Argument 1: Disclosure}
\subsubsection{SE Code 1.04}
\uline{Disclose} to appropriate persons ... any \uline{actual or potential danger} to the user, the public ... that they reasonably believe to be associated with software \cite{se-code}. 

\subsubsection{Actual or potential danger}
% First, define "danger."
The first part of this section to be discussed is the actual or potential danger that results from session hijacking. `Actual' is defined as ``existing in act or fact'' \cite{actual}. `Potential' is defined as ``capable of being or becoming'' \cite{potential}. `Danger' is defined as ``liability or exposure to harm or injury; risk; peril'' \cite{danger}. 

The dangers of session hijacking are described by Butler: ``It's extremely common for websites to protect your password by encrypting the initial login, but surprisingly uncommon for websites to encrypt everything else. This leaves the cookie (and the user) vulnerable. HTTP session hijacking (sometimes called "sidejacking") is when an attacker gets a hold of a user's cookie, allowing them to do anything the user can do on a particular website. On an open wireless network, cookies are basically shouted through the air, making these attacks extremely easy'' \cite{eric-butler}. 

Since Firesheep works by intercepting the ``cookies [that] are basically shouted through the air'' \cite{eric-butler}, any users who are on the same open WiFi network as Firesheep users are potential victims of the dangers associated with non-site-wide HTTPS authentication.

The websites affected by the dangers of HTTP session hijacking at the time of Firesheep's release included websites as large as Google, Facebook and Yahoo! \cite{firesheep-source}. Google's Google+ social network has 343 million users \cite{google-stats}, Yahoo's email service has 281 million users \cite{yahoo-stats}, and Facebook has over 1 billion users \cite{facebook-stats}. Even today, Firesheep works on sites like Stack Overflow, which over 2 million registered users at the time of writing \cite{stack-overflow-stats}.

\subsubsection*{Potential danger: Facebook session hijacking}
Session hijacking has several potential dangers \cite{eric-butler}. At the time of its release, Firesheep allowed users to perform session hijacking on social networks such as Facebook and Google+ \cite{eric-butler}. A session hijacker would take control of your Facebook HTTP session (and thus your Facebook profile) for the duration of that session \cite{eric-butler}. He could post updates to your Wall, your News Feed, or your friends' News Feeds. He could delete friends from your Friends list, or worse, he could record the information of some of your more vulnerable Facebook friends to do any sort of criminal activity he wanted.	

Facebook's News Feed contains timestamped location-tagged photos of your friends in it \cite{facebook-geotag}. Let's say that a session hijacker just hijacked your Facebook session while you were sitting in Starbucks. He sees that your friend John Doe just uploaded (five minutes ago) a picture of his lunch while on vacation in Miami \cite{facebook-geotag}. Two weeks before this Miami vacation, John Doe became the owner of a new dog. He uploaded a picture of this new dog -- this picture was taken two weeks ago and was geotagged at his house \cite{facebook-geotag}. The session hijacker reverse geocodes those coordinates via Google Maps \cite{reverse-geocoding} and determined it to be 258 Chorro Street in California. John is still in vacation in Miami \cite{facebook-geotag}, his house in California is empty \cite{reverse-geocoding}, and the stranger sitting in Starbucks hijacking your Facebook session \cite{eric-butler} is wondering how much of John's stuff he can haul out of John's house while John is in Miami.

\subsubsection*{Potential danger: Stack Overflow session hijacking}
Today, Facebook has correctly authenticated its entire site via HTTPS, but some websites still refuse to fix the issue. Take Stack Overflow for instance \cite{stack-exchange}. Stack Overflow's parent company Stack Exchange still refuses to authenticate their whole website with HTTPS, meaning that Firesheep still works on it today. Stack Overflow is a website for programmers with questions about programming that other programmers can answer \cite{stack-overflow-stats}. The site persists how much help you've given other people over your account's lifetime, and publicly displays how reputable you are in your posts \cite{stack-overflow-stats}. 

Imagine I'm at a hackathon in San Francisco \cite{angelhack} and you're an experienced programmer who also happens to be at this hackathon. We're both on the same open Wi-Fi connection here at this hackathon. We're both Stack Overflow users, but I have the reputation level of a novice, while yours boasts years of good reputation built up from always lending a helping hand. Since we're on the same open Wi-Fi network, I decide to double-click your name on Firesheep \cite{eric-butler}. I've just hijacked your session \cite{eric-butler} thanks to Stack Overflow not authenticating their website through HTTPS \cite{stack-exchange} and now have control of your Stack Overflow account \cite{stack-overflow-stats}. I can do anything I want as you now \cite{eric-butler}. I can distribute incorrect advice to people for recreational purposes \cite{stack-overflow-stats}. I can allocate Stack Overflow `Bounty' to myself \cite{stack-overflow-stats}, rewarding my own Stack Overflow account with the good reputation that you've built up on your account \cite{stack-overflow-stats}. Or, if I get bored, I can just delete your account altogether \cite{stack-overflow-stats}. This could all potentially be done while hijacking your HTTP session via open Wi-Fi at this hackathon, and could still happen today \cite{stack-exchange}.


\subsubsection{Disclose}
To `disclose' something means ``to make known; reveal or uncover'' \cite{disclose}. By definition, this requires that HTTP session hijacking was (relatively speaking) an \uline{unknown} problem before the release of Firesheep.

At the time of Firesheep's release, the users who were most vocal about HTTP session hijacking were in the minority \cite{firesheep-day-later}. Companies like Facebook went years without dealing with the problem, claiming that the problem simply wasn't worth the engineering hours that it would take to fix it \cite{firesheep-day-later}. 

Tools attempting to expose this HTTP session hijacking problem came and went \cite{firesheep-day-later}. ``Very little has changed after each of these tools were released. They got their media hype, and then people forgot or didn�t care. For the most part, the tools were only used by tech-savy people, hackers and geeks'' \cite{firesheep-day-later}. 

Session hijacking had been brought up as an issue within the security community since 2004 \cite{firesheep-day-later}; software companies simply ignored the issue up until Firesheep's release \cite{firesheep-day-later}. Firesheep was another software release aimed to \uline{demonstrate} (disclose) just how vulnerable websites were to HTTP session hijacking \cite{eric-butler}.

\subsubsection{Substituted SE Code 1.04}
Disclose (\uline{demonstrate via software}) to appropriate persons ... any \uline{session hijacking danger} to the user, the public ... that they reasonably believe to be associated with software \cite{se-code}. 

\subsubsection*{Conclusions Drawn}


Eric Butler developed Firesheep as a concrete example of how serious the HTTPS problem was: ``Today at Toorcon 12 I announced the release of Firesheep, a Firefox extension designed to demonstrate just how serious this problem is'' \cite{eric-butler}. That is to say, Firesheep was released specifically with the intent to educate the general public of the security holes many major websites had concerning non-authenticated HTTP sessions. In the blog post accompanying the release of Firesheep, Eric Butler made it very clear that he believed that non-HTTPS session hijacking was a serious problem: ``This is a widely known problem that has been talked about to death, yet very popular websites continue to fail at protecting their users'' \cite{eric-butler}.


To paraphrase SE Code 1.04: `if there's serious danger associated with some software, you're morally obligated to tell the right person what's going on' \cite{se-code}. 

The appropriate persons to disclose this session hijacking problem to are the majority of the general public. To be sure, session hijacking was a known problem to a small subset of the public, and it needed to be disclosed to a much larger audience.

Session hijacking was already possible well before the release of Firesheep through software such as Wireshark \cite{wireshark}. Firesheep itself is actually just an extension of Wireshark \cite{firesheep-source}. Firesheep was so significant simply because it offers a much simpler GUI than traditional session hijacking solutions. % Prove that it really was a "simpler GUI" with factual information!
The release of Firesheep can be seen as a programmer's attempt to `show by doing'; Butler chose to release Firesheep as a catalyst for the seriousness of the HTTPS problem. 

To use Facebook as an example of how cynical companies were about the HTPPS problem: Facebook makes most of their money from  advertising revenue \cite{facebook-revenue}. It follows then that Facebook had no financial incentive to heed the warnings from the small security community (made of engineers like Butler) who were originally so vocal about the issue. With companies as large as Facebook ignoring the issue, the only way to create a change (at the time) was for the issue to come to light to the non-technical majority of the Internet's users.

Therefore, Eric Butler had an ethical responsibility, by SE Code 1.04, to create and release of Firesheep to the general public in order to \uline{disclose} the \uline{dangers of session hijacking} to the public.


% Arugment 2: 
\subsection{Argument 2: Volunteer professional skills}
\subsubsection{SE Code 1.08}
Be encouraged to \uline{volunteer} \uline{professional skills} to \uline{good causes} \cite{se-code}.

\subsubsection{Volunteer}
To `volunteer' something means ``to offer oneself or one's services for some undertaking or purpose'' \cite{volunteer}. Eric Butler wrote Firesheep in his spare time and without pay \cite{eric-butler}, thus offering his services to the community for free. He also open sourced to the public, allowing others extend it by writing their own handlers for it \cite{eric-butler}. His \uline{purpose} for writing Firesheep, as documented in its source code, was to demonstrate HTTP session hijacking vulnerabilities \cite{firesheep-source}.

\subsubsection{Professional skills}
A `professional' is ``a person who is expert at his or her work'' \cite{professional}. `Skills' are ``the ability, coming from one's knowledge, practice, aptitude, etc., to do something well'' \cite{skills}. 

The professional skills, in this case, are the skills Eric Butler used to \uline{write an effective GUI} for Wireshark (the existing session hijacking software) \cite{wireshark}. 

\subsubsection*{An Effective GUI}
% Define an effective GUI using factual data.
Butler states ``Firesheep is doing the exact same thing as these other tools [Wireshark], but with a simpler user interface... Because of its simplicity, Firesheep has already succeeded in demonstrating the risks of insecure websites to a much wider audience than any previous tool, in a single day'' \cite{firesheep-day-later}. 

He backs this claim with numbers: ``Since being released just over a day ago, Firesheep has been downloaded over 129,000 times. Firesheep has consistently been one (if not more) of the `Top Tweets' on Twitter, on top of Hacker News, was at one point the \#10 trending search on Google in the US, and is the second suggestion on Bing when you start typing �fire�. Firesheep has been mentioned on countless blogs and news sites in numerous languages, and has received almost universal praise'' \cite{firesheep-day-later}.

\subsubsection*{Results of Butler's professional skills}
Qualified software hackers used software like Wireshark before the existence of Firesheep \cite{wireshark}. Butler provided his professional skills by integrating Wireshark within Firefox and writing it to be mostly autonomous \cite{eric-butler}. What was once a process reserved for only the most expert of software crackers accustomed to software like Wireshark \cite{wireshark} became a one-click `buddy-list' for session hijackers to choose their victims from \cite{eric-butler}. Overnight, Wireshark turned into Firesheep, with its 129,000 users and multitude of media coverage \cite{firesheep-day-later}.

\subsubsection{A Good Cause}
``\uline{Good cause} is a legal term denoting adequate or \uline{substantial grounds} ... \uline{to take a certain action}, or to fail to take an action prescribed by law. What constitutes a good cause is usually determined on a case by case basis and is thus relative'' \cite{good-cause}.

\subsubsection*{Substantial grounds to take a certain action}
`Good cause', by definition, is calculated on a case by case basis. In the case of Firesheep, Eric Butler believed (and wrote in the release notes of Firesheep) that the actual and potential dangers of HTTP session hijacking compelled him to create and release Firesheep \cite{eric-butler}. The objective evidence of HTTP session hijacking indeed having dangers is presented in Argument 1 of this report. 

\subsubsection{Substituted SE Code 1.08}
[Eric Butler is] encouraged to \uline{volunteer} \uline{his skills writing an effective GUI for Wireshark} to \uline{the good cause of preventing session hijacking}.

\subsubsection{Conclusions drawn}
Butler forcing the hand of the websites responsible for HTTP session hijacking vulnerability to fix their security flaw should be interpreted as `good cause': the resulting HTTPS fix eliminates potential session hijackings from occurring on that website \cite{eric-butler}.

Prior to Firesheep, popular websites were rampantly vulnerable to session hijacking at the time of Firesheep's release \cite{firesheep-source}. The list of vulnerable websites is listed in Section 1.1, and included sites as popular as Google and Facebook. Additionally, ``the risks of insecure websites have been known for years, yet little has been done about what has become an increasingly widespread problem'' \cite{butler-fallout}. 

Victims of session hijacking ultimately have their `happiness' taken from them in exchange for the happiness of the culprit. \uline{Preventing the session hijacking} altogether results in a net gain in happiness of the society and is thus a \uline{good cause}, according to utilitarian ethics \cite{virtue-ethics}. 

Put differently: the release of Firesheep quickly led to the security fix that was so desperately needed in the industry. There was an overhead cost to society of having Firesheep run rampant in the weeks it took for websites to fix this flaw (see Argument 1's analysis). This overhead resulted in a brief loss in happiness to the general public. However, the overall result of Firesheep was that non-HTTPS session hijacking mostly became a thing of the past. The fix of session hijacking resulted in the removal of the dangers of session hijacking to society, and in turn resulted in an increase in  net happiness to the general public. That net positive result accumulates for every year that session hijacking is \uline{not} the problem it was from 2004-2010 \cite{firesheep-day-later}.

The ACM Code of Ethics therefore applies to Eric Butler as he was indeed encouraged to \uline{volunteer} his \uline{GUI skills} to this \uline{good cause}. Similar to Argument 1, the rampancy of vulnerable HTTP websites at the time essentially necessitated Butler's intervention by releasing the Firesheep GUI on top of Wireshark. By this logic, Butler was not only ethical in releasing Firesheep to the general public, but was encouraged to do so on a basis of the utilitarian aspects of the Code of Ethics, Section 1.08.

% Argument 3:
% The METHOD that he used, don't assume that he used an appropriate method. First, define an "appropriate method." For the "accelerate the response" part, be sure to quote that he was doing it for that purpose. As far as "appropriate methods", be sure to list the alternatives that Butler has by proving it with citations. E.g., White Hat will sometimes tell companies about a security defect that they have, and then gives them two weeks to fix it. If they fail to do it, White Hat releases an exploit for the software. Look up CERT for principles on the alternatives.
\subsection{Argument 3: Ensure an appropriate method}
\subsubsection{SE Code 3.05}
Ensure \uline{an appropriate method} is used \uline{for any project on which they work} or propose to work.

\subsubsection{The Project: SSA of the HTTP Protocol}
Butler described in his blog post accompanying Firesheep: 

\subsubsection*{SSA: Software Security Assurance}

\subsubsection*{The HTTP Protocol}


\subsubsection{An appropriate method}
% Define "appropriate method" with respect to White Hat's philosophy on penetration testing.

\subsubsection*{White Hat's philosophy on SSA}


The \uline{appropriate method} to solve this problem was \uline{the release of (controversial) software designed to accelerate the rate of response to HTTP session hijacking vulnerabilities}.

\subsubsection{Substituted SE Code 3.05}
Ensure an appropriate method \uline{(software designed to accelerate the rate of response to HTTP vulnerabilities)} is used \uline{to promote the well-being of the HTTP protocol as a whole}.

\subsubsection{Conclusions drawn}
As was stated in previous arguments in this report, HTTP session hijacking was a problem that was talked about to death among security professionals for years \cite{firesheep-day-later}. That is to say, companies have been outright ignoring this problem for (at least six \cite{firesheep-day-later}) years with no intention of changing their stance at the time of Firesheep's release. For a person in a position like Eric's it would be easy to argue that after six years of fruitless warnings, it was about time to `raise the stakes' a little and force the hands of these non-compliant software companies. If verbal signals for those six years turned out to not be an appropriate method for companies as (evidently) ignorant as this, then a `show by doing' demonstration of the gravity of the HTTP vulnerability was certainly \uline{an appropriate method}.

Following the guidelines of SE Code section 3.05, Butler thus had no choice but to use a more appropriate method to deal with this known issue: he had to raise the stakes a little and release Firesheep to the public. Software engineers under this code have an ethical responsibility to ensure that an appropriate method is used; SE Code 3.05 is yet another perspective on why \uline{Butler was ethical} and ethically obligated to release Firesheep.




%Merriam-Webster dictionary defines \uline{responsibility} as ``the quality or state of being morally, legally or mentally accountable'' \cite{responsibility-definition}. Volunteering your own personal skills to fix a mistake in existing software is just about the most efficient way to hold yourself (or your industry) accountable for said mistake. Not only this, but forcing websites to fix their security bug by releasing Firesheep directly holds the software industry accountable for the HTTP session hijacking issue. 

%As such, Butler's \uline{voluntary development and release of Firesheep} can be interpreted as a very efficient way of taking personal responsibility for the threat to session hijacking that the software industry (as a whole) allowed to exist. More importantly, Firesheep \uline{forced the software industry to take responsibility for its error} at a time when the industry simply did not want to.

%\subsubsection{Errors in software}
%As described in the Abstract, many websites took to authenticating sessions via HTTPS at the moment with the user authentication transaction occurred \cite{security-now}. However, \uline{delivering the rest of the website over regular, unencrypted HTTP} is an \uline{error in software} that left users of the site vulnerable to session hijacking \cite{security-now}. 

%\subsubsection{Substituted SE Code 6.08}
%[The software industry should] take responsibility (\uline{that is, fix the HTTP session hijacking vulnerability}) for detecting, correcting and reporting errors in software (\uline{websites offering non-login pages over regular HTTP}).

%\subsubsection{Argument 3 Analysis}
%Following the theme of Butler's moral responsibility to release Firesheep, the ACM Code of Ethics states that software engineers must take responsibility for errors in software. A website's negligence to authenticate an entire web session via HTTPS is an error that results in user vulnerability to session hijacking, and so it follows that some software engineer must take responsibility for the error \cite{se-code}. 

%Firesheep is Eric Butler's way of both taking responsibility for the error himself (out of ethical responsibility as a software engineer) while also forcing the industry to take responsibility for its error. \uline{Butler was therefore ethical to release Firesheep to the public.}



%%%%%%%%%%%%%%%%
%%% Conclusion %%%
%%%%%%%%%%%%%%%%
%\section{Conclusion}
%The conclusion is a summary of your entire analysis. It should reiterate the answer your audience has been forming while reading your analysis. New information should never be introduced in your conclusion. \cite{texTemp}
\section{Conclusion}
A common theme throughout this discussion has been Eric Butler's ethical responsibility to do \uline{something} about the rampant HTTP session hijacking vulnerability among popular websites. It was, as Eric said, a problem that was talked about to death \cite{eric-butler}. Yet, very few software companies were willing to spend the engineering hours fixing it until Butler trivialized the session hijacking process \cite{firesheep-day-later}. The Code of Ethics describes situations like this in sections 1.04, 1.08, 6.08 and 1.05 \cite{se-code}, and the Code  states that Butler had a moral responsibility to do something about it. His release of Firesheep was  ethical.

Software engineers in similar positions should consider the approach Butler used in garnering attention towards an issue: let your code speak. If software is the cause of some major danger to people, software is also likely to be a solution to said danger. This was especially true in Butler's case, and can be true for many software engineers who are in a position to do something morally sound according to the Code of Ethics. To paraphrase section 1.04 once again: ``if you see some danger associated with some software, \uline{do something} about it'' \cite{se-code}.


%end the two column format
\end{multicols}
\newpage

%cite all the references from the bibtex you haven't explicitly cited
\nocite{*}

\bibliographystyle{IEEEannot}

\bibliography{texreport}

\end{document}
