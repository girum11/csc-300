% CSC 300: Professional Responsibilities
% Dr. Clark Turner

% Two Column Format
\documentclass[11pt]{article}
%this allows us to specify sections to be single or multi column so that things
% like title page and table of contents are single column
\usepackage{multicol}
\usepackage[normalem]{ulem}

\usepackage{setspace}
\usepackage{url}

%%% PAGE DIMENSIONS
\usepackage{geometry} % to change the page dimensions
\geometry{letterpaper}

\begin{document}

\title{\vfill Term Paper Short Draft} %\vfill gives us the black space at the top of the page
\author{
By Girum Ibssa\vspace{10pt} \\
CSC 300: Professional Responsibilities\vspace{10pt} \\
Dr. Clark Turner\vspace{10pt} \\
}
%\date{October 22, 2010} %Or use \today for today's Date
\date{\today}

\maketitle

\vfill  %in combinaion with \newpage this forces the abstract to the bottom of the page
\begin{abstract}
%5\%  One or two sentences of context about your issue. Provide your focus question. Mention one or two relevant arguments.  Briefly explain your conclusion and why it is proper ethically. Should be no more than one paragraph (100 words max) \cite{handout}
Firesheep is an extension for the Firefox web browser that wraps Wireshark (an existing piece of session hijacking and packet sniffing software \cite{wireshark}) in a simple GUI \cite{firesheep-source}. Created by Eric Butler and Ian Gallagher, Firesheep's motivation was described by its creators: ``We're bringing up this tired issue to remind people of the risks they face, especially when on open WiFi networks, and to remind companies that they have a responsibility to protect their users. To drive this point home, we are releasing an open source tool at ToorCon 12 which shows you a `buddy list' of people's online accounts being used around you, and lets you simply double-click to hijack them'' \cite{security-now}.

Was it ethical to release this software? Eric stated in his release of Firesheep that ``It's extremely common for websites to protect your password by encrypting the initial login, but surprisingly uncommon for websites to encrypt everything else...This is a widely known problem that has been talked about to death, yet very popular websites continue to fail at protecting their users'' \cite{eric-butler}. Yet the use of wire sheep may be illegal in the US and beyond \cite{illegal-to-use-firesheep}. I will show that Firesheep was indeed ethical to release due to SE Code 1.04, the requirement for Software Engineers to fully disclose any potential software danger to the public \cite{se-code}.

\end{abstract}

\thispagestyle{empty} %remove page number from title page
\newpage


%Create a table of contents with all headings of level 3 and above.
%http://en.wikibooks.org/wiki/LaTeX/Document_Structure#Table_of_contents has
%info on customizing the table of contents
\thispagestyle{empty}  %Remove page number from TOC
\tableofcontents

\newpage

%end the 1 column format


%start 2 column format
\begin{multicols}{2}
%Start numbering first page of content as page 1
\setcounter{page}{1}
%%%%%%%%%%%%%%%%%%%%
%%% Known Facts  %%%
%%%%%%%%%%%%%%%%%%%%
\section{Facts}
% Known facts that are not disputed that lead to your question. Do not judge these facts or make anything like an argument for an answer in here. Just note the facts that give us the general background and end them with the facts leading to the controversy you are interested in. The reader should naturally be asking the question you'll be asking by that point in your paper. In general, attach your facts to a specific case, the more specific and detailed the facts, the better for your analysis. Cite all facts to their sources. \cite{handout}

%\begin{itemize}

%\item General context of the issue
%\item Only facts that clearly point to the issue/question at hand (you may add additional facts as needed in your analysis that aren't in your facts section)
%\item The facts section should end up leading the reader to the question you are about to ask them (a controversy or a case ending poorly is a great way to do this).
%\item Should not indicate anything about your answer to the question or hint at any particular conclusion.
%\cite{handout}
%\end{itemize}

\subsection{Description of Firesheep}
Eric Butler describes in his blog: ``When logging into a website you usually start by submitting your username and password. The server then checks to see if an account matching this information exists and if so, replies back to you with a `cookie' which is used by your browser for all subsequent requests \cite{eric-butler}.

``It's extremely common for websites to protect your password by encrypting the initial login, but surprisingly uncommon for websites to encrypt everything else. This leaves the cookie (and the user) vulnerable. HTTP session hijacking (sometimes called ``sidejacking'') is when an attacker gets a hold of a user's cookie, allowing them to do anything the user can do on a particular website. On an open wireless network, cookies are basically shouted through the air, making these attacks extremely easy'' \cite{eric-butler}. Butler continues with his description of the ``problem that has been talked about to death'' and how it remains to be solved \cite{eric-butler}. 

Qualified hackers were already able to perform session hijacking like this well before the release of Firesheep, using the program `Wireshark' (which Firesheep is built on top of). \cite{wireshark}.

Firesheep works by allowing users to write custom "handlers" for it to allow it to work on webpages of their choice. Firesheep's source code currently includes "handlers" for several popular websites including: Amazon, Basecamp, bit.ly, Enom, Facebook, FourSquare, Github, Google, Hacker News, Harvest, The New York Times, Pivotal Tracker, Twitter, ToorCon: San Diego, Evernote, Dropbox, Windows Live, Cisco, Slicehost, Gowalla, Flickr and Yahoo \cite{firesheep-source}. 

Some of those websites have quickly fixed the security hole (not using HTTPS) that Firesheep relies on to work \cite{butler-fallout}. However, a strong portion of websites simply didn't fix the security flaw, weeks and months after the release of Firesheep \cite{butler-fallout}.


% (Clark told me to take out this section) \subsection{Definition of the ACM Code of Ethics}
%The Association for Computer Machinery defines a Code of Ethics stating that ``software engineers shall commit themselves to making the analysis, specification, design, development, testing and maintenance of software a beneficial and respected profession'' \cite{se-code}. 

% (Clark told me to take out this section) \subsection{Firesheep's adherence to said Code of Ethics}
%Firesheep's release was intended to allow the general public perform session hijacking on suspecting or non-suspecting victims, with consequences that may not be ``beneficial and respected.'' However, Firesheep's release prompted many large software companies to quickly fix the security holes that Firesheep was designed to expose \cite{disconnect-blog}; the companies were slow to fix it before Firesheep's release \cite{disconnect-blog}. From a strictly objective point of view, Firesheep was detrimental to public welfare in the short term and was beneficial to public welfare in the long term.



%%%%%%%%%%%%%%%%%%%%%%%%%
%%% Research Question %%%
%%%%%%%%%%%%%%%%%%%%%%%%%
%One sentence (one question, not compound). Should be focused on a particular case. Should have a determinable yes or no answer that you will draw based on your research.  Should be followed (separately) by a paragraph explaining the importance of this question and its relevance to software engineers (why should we care?). \cite{handout}

\section{Research Question}
% Your research question -- this is the ethical question you are interested in answering. It should be one simple sentence and lead to a yes/no answer. It needs to be very narrowly focused, specific, and not abstract at all. It's best to question a detailed case in the general area of your interest. Open ended questions are very hard to answer. \cite{handout}

Was it ethical for Eric Butler to release Firesheep to the public, knowing that it could be used for harm in the hands of the wrong individuals?

\subsection{Relevance}
Firesheep can be downloaded today ({\today}) from GitHub.com, a website hosting free open source programs \cite{firesheep-source}. At its time of release, several websites simply refused to implement HTTPS site-wide, allowing Firesheep to work on a large number of sites \cite{eric-butler}. Today, Firesheep still works on a few sites that haven't switched to using HTTPS site-wide, including the entire Stack Exchange network of websites \cite{stack-exchange}.

Session hijacking allows users of Firesheep to impersonate "logging in" as their victims on the sites that it works on \cite{eric-butler}. The consequences of this can be anything from posting false Facebook status updates to outright deleting a person's Stack Overflow profile. Eric Butler's free software simplifies this process to the point where normally unqualified people may perform these acts, increasing the vulnerability of typical users worldwide.


%%%%%%%%%%%%%%%%%%%%%%%%%
%%% Extant Arguments from External Sources %%%
%%%%%%%%%%%%%%%%%%%%%%%%%

%\section{Arguments For}
%\subsection{Arg 1}
%The first argument for your topic
%\subsection{Arg 2}
%The second argument for your topic...
%\section{Arguments Against}
%\subsection{Arg 1}
%The first argument against your topic.
%\subsection{Arg 2}
%The second argument against your topic...

%the * causes a section with no numbering also doesn't appear in the table of contents
%\subsubsection*{Requirements for the Arguments section(s) (from the handout)}
%Summarize the main arguments others have made about the answers to your focus question. Provide the state of research on your focus question. Must be referenced appropriately.  All statements must be a summary of the source's arguments, devoid of your opinions or biases on the issue. Must (separately) cover arguments on both sides of your issue - that is, those that answer your focus question affirmatively and those that answer negatively. \cite{handout}

\section{Extant arguments}
% Extant arguments -- this is where you gather the arguments made by others interested in the same question. No judgments, just repeat their arguments for the answer in the best possible light from the arguer's perspective. Cover both sides of your question (the ``yes'' side and the ``no'' side) to get a complete picture of how others are thinking about it. Do not include any general ethical principles in here unless they are explicitly written up in the arguments. Cite all arguments to their sources. 

\subsection{In Favor of Firesheep's release}
\subsubsection{Code as "free speech"}
Eric Butler himself argues in favor of the release of Firesheep. In his article ``Firesheep, a week later: Ethics and Legality'', Butler states outright that ``it is nobody's business telling you what software you can or cannot run on your own computer'' \cite{butler-week-later}. He defends by saying that code is a form of free speech, and that we have a Constitutional right to free speech \cite{butler-week-later}.

\subsubsection{Mozilla's support of Firesheep}
Mozilla themselves support Firesheep. Firefox (the browser for which Firesheep is an extension for) features an internal blacklist of extensions that it does not allow to work \cite{mozilla-blocklist}. Mozilla, the creators of Firefox, specifically decided not to blacklist Firesheep \cite{no-kill-switch}. Mike Beltzner, director of Firefox, praised its release: ``[Firesheep] demonstrates a security weakness in a number of popular websites, but does not exploit any vulnerability in Firefox or other Web browsers'' \cite{no-kill-switch}.


\subsection{Against Firesheep's release}
\subsubsection{Real-world use of Firesheep may be illegal}
The actual use of Firesheep, however, may be illegal \cite{illegal-to-use-firesheep}. Federal wiretapping laws state that it's not illegal ``to intercept or access an electronic communication made through an electronic communication system that is configured so that such electronic communication is readily accessible to the general public'' \cite{illegal-to-use-firesheep}. However, Jonathan Gordan, partner at Los Angeles law firm Alson and Bird, states that ``when people are accessing their social network [account], they have an expectation that whatever they're doing is governed by the privacy settings in that network'', and that a open Wi-fi network does not qualify as ``readily accessible to the general public'' \cite{illegal-to-use-firesheep}.

\subsubsection{Firesheep desired to violate privacy}
Firesheep is built specifically to allow untrained users to hijack browser sessions from unsuspecting victims \cite{eric-butler}, which directly contradicts Section 1.03 of the Code of Ethics: ``1.03. Approve software only if they have a well-founded belief that it is safe, meets specifications, passes appropriate tests, and does not diminish quality of life, diminish privacy or harm the environment'' \cite{se-code}. 



%%%%%%%%%%%%%%%%
%%% Analysis %%%
%%%%%%%%%%%%%%%%
\section{Analysis}
%\begin{itemize}
%   \item Should start with a paragraph showing why the SE Code applies to your focus
%question.
%   \item Sub-headings to delineate your lines of reasoning are required.
%   \item All arguments must be thoroughly supported by reason and logic.
%   \item All claims must be supported by reputable primary sources and formal data.
%   \item SE Code must be central to the argumentation
%   \begin{itemize}
%      \item You should have 2-4 distinct sections of the SE code utilized in your analysis
%      \begin{itemize}
%         \item If section 1 is central to your argument, it is only one of the code sections covered. Do not rely solely on section one. Ex: 1.01-1.04 will not suffice for all of your SE Code based arguments and citations.
%         \item If discussion about Òpublic goodÓ is used, there must be data to support it. Simply writing Òit benefits the general public because it would make many people happyÓ is insufficient.
%      \end{itemize}
%   \end{itemize}
%   \item Utilitarian and deontological analysis must be present but not be separate sections
%   \item Class reading must be referenced as appropriate (at least one paper must be used as the basis of one of the arguments).
%   \item There should be a clear cohesiveness to the analysis such that each argument logically flows into the next and gently directs the reader toward your conclusion while implicitly providing answers to any doubts they may have through logic and data.
%   \item Opinions > dev/null. \cite{handout}
%\end{itemize}

%Look at Jason Anderson's how to write a term paper (currently linked as the paper template) for information on how to write this section.  An example of possible sections follows
%\subsection{Why the SE Code Applies}
%\subsection{Argument 1}
%\subsubsection{Code principle 1 that applies}
%\subsubsection{Code principle 2 that applies}
%\subsection{Argument 2}
%\subsubsection{Code principle 1 that applies}
%\subsubsection{Code principle 2 that applies}

%\subsubsection*{}
%Remember to weave the class papers and other ethical systems arguments in with the se code arguments they shouldn't be separate sections.

\subsection{Why is the SE Code of Ethics applicable to this problem?}
The Code defines software engineers as ``those who contribute by direct participation to the...design, development...of software systems'' \cite{se-code}. Did Eric Butler contribute by ``direct participation'' to the design of some ``software system''?

Firesheep is the name of software written in the C++ programming language that wraps existing packet sniffing software (Wireshark) in an easy to use, one-click GUI extension for the Firefox browser \cite{firesheep}. Firesheep is a software system directly written by Eric Butler, maintained on his open source GitHub account \cite{firesheep-source}. Eric Butler (the software engineer) has therefore contributed by direct participation (programming himself) in the design of a ``software system'' (Firesheep) \cite{firesheep}.

Since he contributed by ``direct participation'' to ``the design'' of a ``software system,'' Eric Butler is defined to be a ``software engineer'' under the ACM Code of Ethics, and therefore must follow the Code \cite{se-code}.

\subsection{Principle 1: Disclosure}
\subsubsection*{SE Code 1.04}
Disclose to \uline{appropriate persons} ... any \uline{actual or potential danger} to the \uline{user, the public} ... that they reasonably believe to be \uline{associated with software} \cite{se-code}. 

\subsubsection*{Substituted SE Code 1.04}
\subsubsection*{SE Code 1.04}
[Eric Butler shall] Disclose to \uline{the open-source public} ... any \uline{session hijacking dangers} to the \uline{users} ... that they reasonably believe to be \uline{associated with non-site-wide HTTPS authentication} \cite{se-code}. 

\subsubsection{}
Firesheep was released specifically with the intent to educate the general public of the security holes many major websites had concerning non-authenticated HTTP sessions \cite{eric-butler}.



%%%%%%%%%%%%%%%%%%%%%%%%
% Clark says *don't* use principle 6.06. %%%
%%%%%%%%%%%%%%%%%%%%%%%%
%\subsection{SE Principle 6.06}
%\subsubsection{Definition}
%``6.06. Obey all laws governing their work, unless, in exceptional circumstances, such compliance is inconsistent with the public interest'' \cite{se-code}.
%\subsubsection{Pertinence}
%Laws in the US, UK and beyond generally outlaw the use of Firesheep for most of its use cases \cite{illegal-to-use-firesheep} \cite{illegal-to-use-firesheep-uk}. However, Firesheep's release managed to force software vendors to quickly fix the security holes that Firesheep was designed to expose \cite{disconnect-blog}. These software vendors were notoriously slow to fix the security hole up until the release of Firesheep \cite{disconnect-blog}. 


% (Clark says hold off on this principle for the Short Draft) \subsubsection{SE Principle 1.03}
%``1.03. Approve software only if they have a well-founded belief that it is safe, meets specifications, passes appropriate tests, and does not diminish quality of life, diminish privacy or harm the environment. The ultimate effect of the work should be to the public good'' \cite{se-code}.

%Firesheep's allows otherwise unqualified hackers to perform session hijacking on unsuspecting victims \cite{eric-butler}. Increasing the size of the pool of people qualified to perform a crime clearly diminishes privacy and quality of life in the short term.

% (Clark says hold off on this principle for the Short Draft) \subsubsection{SE Principle 1.08}
%``1.08. Be encouraged to volunteer professional skills to good causes and contribute to public education concerning the discipline'' \cite{se-code}.

%Firesheep was developed in Eric Butler's free time, released in the hope that a simpler UI would give greater exposure to a long-known security vulnerability in several major websites \cite{firesheep-huge-hit}. It succeeded in this purpose \cite{firesheep-day-later}.


%\subsubsection{A Deontological Perspective}
% a) start with deontological perspectives as a section where you analyze those arguments based on the inherent ethics of the act itself rather than the results or trade-offs;
%\subsubsection{`Deontological' Definition}
%Deontological ethics is a normative ethical position that judges the morality of an action based on whether or not the action adheres to existing rules or laws \cite{virtue-ethics}.

%\subsubsection{Relevant SE Code Article}
%The article of the SE Code of Ethics that is most pertinent to a Deontological perspective of ethics is Article 6.06: ``6.06. Obey all laws governing their work, unless, in exceptional circumstances, such compliance is inconsistent with the public interest'' \cite{se-code}. The first half of this article is a Deontological perspective into the ethics of software engineering. The use of Firesheep in certain situations, as stated before, is illegal and thus shouldn't be used. 




%\subsubsection{A Utilitarian Perspective}
% then, b) use a utilitarian perspective and list the appropriate analyses of the trade-offs and stakeholders to define the most desired results and how to get them. Be explicit about the trade-offs (what value is balanced against what other value, which stakeholders win, which stakeholders lose...) What is the ``utility'' in ``utilitarian'' in your case -- what value do you want to advance the most (derived from the general utilitarian ``happiness'')? How do you maximize (or optimize) it?
%\subsubsection{`Utilitarian' Definition}
%Utilitarian ethics are defined such that ``the rightness or wrongness of a particular action is a function of the correctness of the rule of which it is an instance'' \cite{rule-utilitarianism}.

%\subsubsection{Relevant SE Code Article}
%The second half of SE Code Article 6.06 states you should adhere to rules ``unless, in exceptional circumstances, such compliance is inconsistent with the public interest'' \cite{se-code}. The latter part of this article is a Utilitarian perspective on the ethics of software engineering \cite{rule-utilitarianism}. 

%The length for which session hijacking went as an unsolved problem necessitated the release of Firesheep as an `exceptional circumstance.' Firesheep was thus ethical to release with respect to a Utilitarian perspective on SE ethics (that is, Article 6.06 of the SE Code) \cite{se-code}.



%%%%%%%%%%%%%%%%
%%% Conclusion %%%
%%%%%%%%%%%%%%%%
%\section{Conclusion}
%The conclusion is a summary of your entire analysis. It should reiterate the answer your audience has been forming while reading your analysis. New information should never be introduced in your conclusion. \cite{texTemp}

%end the two column format
\end{multicols}
\newpage

%cite all the references from the bibtex you haven't explicitly cited
\nocite{*}

\bibliographystyle{IEEEannot}

\bibliography{texreport}

\end{document}
